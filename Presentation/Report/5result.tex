\subsection{User Experience}
\subsubsection{Login / Register}
For the login we choose to use the email as the users unique identifier and a password for the means of authentication. The main reason for going with a username and password authentication instead of a oauth solution is because this solution can be used without having a third party account. Implementing an oauth solution that everyone can use is time consuming because you would need to have more than one implementation (like Google+, Facebook or Twitter login methods) and therefore not fitting for the time scope of the project. This means that a username and password system would need to be implemented in case a user has no accounts on the given services. The main drawback with this kind of login method is that it requires the user to remember a new password instead of using an already existing username and passphrase. 
\subsubsection{Friends Management}
To make the process of inviting friends to the game more easy we choose to implement a friends list feature.\\ 
Friends are added to the friends list by entering their email address and after that they can be invited to any lobby by just a single click or tap. Our implementation does not require the other party to accept the friends request (like in a social network) and the major reason for that is because our friends list is more suppose to behave like a shortcut for inviting friends to the game. Having the other party accepting a friends request is therefore unnecessary since our application doesn't limit how users can send invitations to each other. This is however a feature that needs to be considered if the number of social features increase in the application (like a chat feature for instance).
\subsubsection{Lobby}
The lobby was chosen to allow users to make a party of friends before going into a game. 
\bigskip
The owner of a lobby may choose to invite how many friends he/she wishes for, however there are only 2 or 4 slots that may be filled (the owner takes one slot), how many slots there are depends on the game type. The remaining users may not join the lobby when it has become full. The owner may however start the game at any time, in such an occasion the lobby will become locked from joining by invitation.\\ 
The lobby will then be placed in a queue depending on game type and how many players are currently in the game. The lobby then becomes merged with other lobbies until such a time that all slots are filled, the lobby will then be sent to the game service to become a game.
\subsubsection{Game}
When a game starts it will be added to the games list where it can have two different kinds of states, waiting for other players and ready to place brick.\\
The game board itself is represented by bricks which is placed in an eight by eight square. White bricks are untaken places on the board and the players bricks are represented in different colors. The board view also includes a list showing which colors the players have.\\
When a user wants to place a brick on the board he/she first picks a position on the board, if it's a legal move the user will be prompted with a question. In this state the user can either answer correctly or incorrectly to the question and also time out if a answer has not been given after 20 seconds. When a question times out it will act as a false answer. If more than one player has chosen the same brick and answered correctly to the question there will be a "Question Battle" to determine which player gets the brick. In a question battle the players will each get a new question, if a player answers incorrectly to a question they are out. If more than one player answers correctly to a question they will all be given a new question. The question battle continues until there are either one player or no one left. In the case of the second outcome (where all of the users has answered incorrectly to the current question) no one will receive the brick and it will be left untaken in the second round.\\
If a player has answered the question correctly the brick on the chosen position will either get their color or a lighter version of it. The second option acts as a placeholder when there still are players in the game that haven't picked a place to place their bricks. This is to to give a visualisation to the placing of the brick and to inform the player that it might be a question battle over it.\\ 
When all the players have placed their bricks and all the question battles are over a new round will begin.
\subsubsection{Cross Platform}
The application created for this project is available on a total of three platforms and a user can switch between them. All of the platforms have the same features so you don't lose any part of the core experience when on one of the applications. The big difference between all of the clients is that they have different user interfaces as we chose to follow the platform guidelines for user interfaces.\\
Because the state of the game is controlled by the server it means that a user can log in to any of the three devices and resume their game where they left of. This gives the users more freedom in how they chose to interact with the application, whether it being on a big screen pc or a portable device on the go. It also means that a user can, for instance, start a new game on the web browser at home and play the next round on the daily commute on the bus.

\subsection{Project Goals}
\subsubsection{Scalability}
Through the use of APIs such as Gevent and Akka, we have provided a scalable sub-architecture inside each process that allows the backend to efficiently utilize available computing resources to serve a very large number of clients simultaneously. Therefore, we feel that we have achieved our initial goal to provide the necessary level of scalability for a project of this character.
\subsubsection{Extensibility}
Since we have purposely divided the backend into subcomponents implemented as processes and have provided an infrastructure that is constructed with extensibility in mind, we have provided ample and powerful means to readily expand the backend with new processes and modules. We deem that we have satisfied our objective of being able to easily extend the system with new functionality.